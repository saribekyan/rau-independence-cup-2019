\problemname{Renovation}

To renovate his home, Armen has bought $n$ tools which he will put in his garage.
The garage is long and narrow, with the entrance at one end.
We will represent the garage as the $x$ axis, with the door being at coordinate $0$.
He can place the tools at positive integer coordinates, only one tool per location.
While Armen is working he always has exactly one tool with him, and the remaining $n - 1$ tools are in the garage.
When he comes to the garage to get a new tool, he enters at coordinate 0, walks directly to where his required tool is, swaps it with the old tool, and walks back to the entrance.
This means that the tools move in the garage throughout the renovation.

Armen knows exactly in what order he will use the tools (some tools may get used more than once).
He wants to know what is the minimal possible distance that he needs to walk in the garage to finish he renovation, if he arranges his tools optimally at the start.

\section*{Input}
The first line of the input contains two integers, $n$ and $m$ ($1 \leq n, m \leq 10^5$), the total number of tools, and the length of his renovation.
The next line of the input contains $m$ space-separated integers between $1$ and $n$.
Armen uses the tools in the order given by this sequence.
At the start of the work, Armen already has the first tool of the sequence in his hand.
It is guaranteed that the sequence does not have the same tool listed consecutively.

\section*{Output}
The only line of the output contains the least possible distance that Armen needs to walk in his garage to finish the renovation.

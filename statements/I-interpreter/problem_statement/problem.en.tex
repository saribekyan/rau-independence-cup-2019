\problemname{Interpreter}

After mastering mathematics and gardening, Armen wants to learn programming.
He wants to start from the basics and learn Basic.
He does not find it easy, and has asked you to write an interpreter for a simple version of Basic prgramming language.

Below is the formal description of a language:
\begin{verbatim}
    var := 'a' .. 'z'
    int := '0' .. '65535'
    arg := <var> | <int>
    op := '+' | '-' | '*' | '/'
    comp := '<' | '=' | '>'
    test := <arg> <comp> <arg>
    expr := <arg> | <arg> <op> <arg>
    command := 'PRINT' <arg> |
               'GOTO' <int> |
               'END' |
               'IF' <test> 'THEN' <command> | 
               <var> = <expr>
\end{verbatim}

The program consists of $n \geq 1$ lines, numbered from $1$ to $n$.
The string on each line is an instance of a \texttt{command}, defined formally above.
The last command is always \texttt{END}.\hnote{Why is this necessary? As long as the program reaches END somewhere, we do not need END at the end.} \enote{Anyway let's keep this line.}
Commands are executed sequentially starting from first command unless a \texttt{GOTO} command is encountered.

Below is the description of the commands:
\begin{description}
    \item[\texttt{PRINT <arg>}] -- prints the value of the constant or variable, followed by a line break.
    It is guaranteed that the value of the variable is defined at the point when it is being used.
    
    \item[\texttt{GOTO <int>}] -- continue the execution to the command starting from the given line number.
    It is guaranteed that the argument of a command is natural number between $1$ and $n$.
    
    \item[\texttt{IF <test> THEN <command>}] -- if after substituting the values of the variables in \texttt{test}, the statement becomes true, then the given command is executed.
    Otherwise, the next command is executed.
    It is guaranteed that the values of variables presented in the statement \texttt{test} are defined.
    
    \item[\texttt{END}] -- the command that finishes the execution of the program.
    After this command the interpreter stops.
    
    \item[\texttt{<var> = <expr>}] -- value of the expression \texttt{expr} is calculated and assigned to the variable \texttt{var}.
    It is guaranteed that the values of the variables in \texttt{expr} are defined.
\end{description}

\section*{Input}
The first line of the input contains an integer $n$, ($1 \leq n \leq 1000$), the number of commands.
The next $n$ lines contain the source code of the program starting from the first command.

It is guaranteed that
\begin{itemize}
    \item all elements of commands are separated by a space;
    \item all division operations are integer divisions, and the remainder is discarded;
    \item after any arithmetic calculations the value of variables will be between 0 and 65535, and there will be no division by 0;
    \item the program is syntactically correct;
    \item that all commands contain no more than 60 symbols;
    \item the program will finish the execution at a command \texttt{END} after no more than 10000 commands.
\end{itemize}

\section*{Output}
The output should contain the result of the execution of the program given in the input, i.e.~what the program will print to the output.


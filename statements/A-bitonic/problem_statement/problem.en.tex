\problemname{Bitonic Skyline}

The city has $n$ buildings, which are arranged along the coast on one line and numbered from $1$ to $n$.
The King wants to improve the view for the visitors from the sea, by making the city’s skyline \emph{bitonic}.
The skyline is called bitonic if, when viewing from the sea from left to right, the heights of buildings strictly increases until the tallest building, and then strictly decreases.
The King does not want to remove floors from the buildings, but he can add some floors.
Of course, adding a new floor to any building is expensive.
The King has asked you to calculate the smallest total number of floors that can be added to the buildings so that the skyline becomes bitonic.
We assume that all the floors in all buildings have the same height.

\section*{Input}
The first line of the input contains the integer $n$, the number of the buildings in the city.
The next line contains $n$ space-separated integers.
The $i$th number corresponds to the number of floors in the $i$th building when viewed from the sea.
No building has more than $10^9$ floors. 

\section*{Output}
The output contains a single integer, the smallest number of floors that the King needs to add to make the skyline bitonic.
\problemname{Scavenger Hunt}

Your friend Armen is playing a game of scavenger hunt.
The game takes place on the horizontal $x$ axis, and initially Armen is at coordinate $0$.
In one second he can move one coordinate to the left or to the right.
The organisers of the game place $n$ treasures, at integer coordinates $x_1,\dots, x_n$, respectively.
These coordinates are unknown to you and Armen.
Initially, Armen is given the description of treasure 1, and when he reaches $x_1$, he will recognise it.
At that point he will be given the description of treasure 2, but, once again, not its coordinate.
Then his goal would be to find treasure 2.
When he reaches treasure 2, he is given the description of treasure 3, and so on, until he reaches all the treasures.

The judges assess his performance based on how long it took him to find the items, compared to the minimum possible time, $T_{min} = \sum_{i = 0}^{n-1}|x_{i + 1} - x_i|$, where $x_0 = 0$.
He receives a prize if he collects all the treasures in at most $8 \cdot T_{min}$ seconds.
Write a program that will give instructions to Armen so that he wins the prize.

\section*{Input and Output}
This is an interactive problem.
Your program will need to print instructions to Armen and, after each instruction, receive an input from Armen telling you whether he has reached the next treasure.
First, your program must read one line containing an integer, the number of treasures $n$ ($100 \leq n \leq 1000$).
Then it needs to print instructions to reach treasure 1.
Each instruction must be one integer $h$, printed on a separate line.
It means that you are instructing Armen to move from his current position $x$ to $x + h$.
After each instruction, your program must read the feedback from Armen in one line.
The line will be either ``\texttt{Yes }$x$'' or ``\texttt{No }$x$''.
This will indicate whether Armen found the treasure, and also give you his current coordinate $x$.
Once you receive a ``\texttt{Yes }$x$'' from Armen, you need to give him instructions to find the second treasure, and so on until the last treasure.
Note that when Armen finds the treasure during the instruction, he stops at the treasure.

It is guaranteed that no treasure is more than $10^9$ seconds away from the origin, thus, you are not allowed to give Armen instructions that will make him go farther.
Below is an example interaction, for $n = 2$ to save space.
In actual tests $100 \leq n \leq 1000$.

{
\renewcommand{\sampleinputname}{Armen's feedback}
\renewcommand{\sampleoutputname}{Your instructions}
\displaysample{../problems/C-scavenger/sample-interaction}
}

Note that, Armen does not need to finish the third instruction because he finds the first treasure at location 10 before finishing it.

